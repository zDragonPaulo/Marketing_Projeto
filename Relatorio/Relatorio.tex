\documentclass[a4paper]{article}
% Pacotes necessários
\usepackage[portuguese]{babel}
\usepackage[backend=biber, style=apa, citestyle=apa, language=portuguese]{biblatex}
\usepackage{csquotes}
\addbibresource{Recursos/referencias.bib}

\usepackage{amsmath}
\usepackage{graphicx}
\usepackage{subcaption}
\usepackage{setspace}
\usepackage{siunitx} % Required for alignment
\sisetup{
  round-mode          = places, % Rounds numbers
  round-precision     = 2, % to 2 places
}
\usepackage{enumerate}
\usepackage{enumitem}
\usepackage{amsmath}
\usepackage{karnaugh-map}
\usepackage[section]{placeins}
\usepackage{geometry}
\usepackage{amssymb}
\usepackage{titling}
\usepackage[T1]{fontenc}
\usepackage{float}
\usepackage[hidelinks]{hyperref}
\usepackage{xcolor}
\usepackage{indentfirst}
\usepackage{array}
\usepackage{soul}
\usepackage{afterpage}
\newcolumntype{P}[1]{>{\centering\arraybackslash}p{#1}}
\onehalfspacing


% Comando para criar uma página vazia
\newcommand\myemptypage{
    \null
    \thispagestyle{empty}
    \addtocounter{page}{-1}
    \newpage
}

% Página de título principal
\newcommand{\firsttitlepage}{
    \begin{titlepage}
        \centering
        \vspace*{1cm}
        
        % Logos superior
        \begin{figure}[h!]
            \centering
            \includegraphics[width=6cm]{Recursos/LOGO_IPB} % Substitua pelo caminho da imagem
            \vspace{0.5cm}
        \end{figure}

        % Informações da instituição
        \large\textbf{INSTITUTO POLITÉCNICO DE BEJA} \\
        \large\textbf{Escola Superior de Tecnologia e Gestão} \\
        \large\textbf{Licenciatura em Engenharia Informática} \\
        \large\textbf{Marketing e Empreendorismo} \\
        
        \vspace{1.0cm}
        
        % Título do projeto
        {\Huge \textbf{EzPower™}} \\
        
        \vspace{1.5cm}
        
        % Autores
        \large João Eduardo Sara Sousa - 23916 \\
        \large Martinho José Novo Caeiro - 23917 \\
        \large Paulo António Tavares Abade - 23919 \\

        \vfill
        
        % Logo inferior
        \begin{figure}[h!]
            \centering
            \includegraphics[width=6cm]{Recursos/IPBejaESTIG.jpg} % Substitua pelo caminho da imagem
        \end{figure}
        
        
        % Local e data
        {\large Beja, abril de 2025}
    \end{titlepage}
}

\newcommand{\secondtitlepage}{
    \begin{titlepage}
        \centering
        \vspace*{1cm}
        
        % Informações da instituição
        \large\textbf{INSTITUTO POLITÉCNICO DE BEJA} \\
        \large\textbf{Escola Superior de Tecnologia e Gestão} \\
        \large\textbf{Licenciatura em Engenharia Informática} \\
        \large\textbf{Marketing e Empreendorismo} \\
        
        \vspace{2cm}
        
        % Título do projeto
        {\Huge \textbf{EzPower™}} \\
        
        \vspace{1.5cm}
        
        % Autores
        \large João Eduardo Sara Sousa - 23916 \\
        \large Martinho José Novo Caeiro - 23917 \\
        \large Paulo António Tavares Abade - 23919 \\

        \vspace{2cm}

        % Orientador
        \large Orientador: Eunice Duarte\\
        
        \vfill
        
        % Local e data
        {\large Beja, abril de 2025}
    \end{titlepage}
}

\begin{document}


\pagenumbering{gobble} % Oculta numeração da página

% Primeira página de título
\firsttitlepage

\secondtitlepage


% Abstract
\section*{\LARGE\textbf{\textit{Resumo}}}

Um produto imperdível para quem procura uma solução de qualidade e eficiência.


\vspace{1cm}
% Keywords
\textbf{Keywords:} rgdp
\newpage
%--------------------------------------------------------------------------------------------------------------------------------------

\section*{\LARGE\textbf{\textit{Abstract}}}

A must-have product for those looking for a quality and efficient solution.



\vspace{1cm}
% Keywords
\textbf{Keywords:} rgdp
\renewcommand{\contentsname}{Índice}       % Título do sumário
\renewcommand{\listfigurename}{Índice de Figuras} % Título da lista de figuras

% Início do conteúdo do relatório
\newpage
\doublespacing
\tableofcontents
\listoffigures
\doublespacing

\newpage
\pagenumbering{arabic}

\section{Introdução}\label{intro}
Neste projeto irá ser abordado como foi o desenvolvimento deste produto e como foram separadas as tarefas para cada membro do grupo.
O produto em questão é uma tomada inteligente que tem como objetivo facilitar a vida dos utilizadores, permitindo o controlo
remoto de dispositivos eletrónicos através de um aplicativo móvel. A tomada é equipada com tecnologia Wi-Fi e Bluetooth,
permitindo a conexão com smartphones e tablets.
%---------------------------------------------------------------------------------------------------------------------------
\section{Desenvolvimento}\label{etl}
Nesta parte do relatório será abordado o desenvolvimento do projeto, desde a sua preparação, desenho e forma final. Será também
abordada a separação de tarefas entre os membros do grupo, bem como a forma como cada um contribuiu para o projeto.
%---------------------------------------------------------------------------------------------------------------------------

\subsection{Metodologia de Trabalho}\label{metodologia}
A metodologia de trabalho utilizada foi uma variação do SCRUM (\cite{SCRUM}), onde cada membro do grupo tinha tarefas específicas a realizar.
Foram realizadas reuniões pela plataforma \textit{Discord} (\cite{discord}) eram feitas com variações de 1 a 2 dias de intervalo,
onde cada membro do grupo apresentava o que tinha feito e o  que iria fazer a seguir. A confirmação da finalização das tarefas
era feita através de e-mails, onde cada membro do grupo  utilizava o seu e-mail institucional para informar os restantes
membros do grupo e o professor orientador.
%---------------------------------------------------------------------------------------------------------------------------
\subsection{Separação de Tarefas}\label{separacao}
A separação de tarefas foi feita de forma a que cada membro do grupo tivesse tarefas específicas a realizar, sendo que ficou dividido
da seguinte forma:
\begin{itemize}
	\item João Eduardo Sara Sousa - 23916: Parte X, Y, Z do Excel.
	\item Martinho José Novo Caeiro - 23917: Interpretação do Excel e Parte X, Y e Z do Excel.
	\item Paulo António Tavares Abade - 23919: Relatório e Parte X, Y e Z do Excel.
\end{itemize}
%---------------------------------------------------------------------------------------------------------------------------
\subsection{Desenho do Produto}\label{desenho}
O desenho do produto foi feito pelo membro do grupo Martinho José Novo Caeiro - 23917, sendo que o mesmo utilizou um desenho
feito à mão e utilizou o \textit{ChatGPT} (\cite{chatgpt}) para o ajudar a criar o desenho do produto.
O desenho foi feito com o objetivo de ser o mais simples possível, de forma a que o utilizador consiga perceber como funciona o produto e como
que funcionalidade é que teria. Após uma conversa do grupo, o desenho foi alterado para que fosse ainda mais inovador e agregasse mais valor ao
produto, adicionando tecnologias que não estavam previstas inicialmente como o \textit{Bluetooth} e o \textit{Wi-Fi}.
%---------------------------------------------------------------------------------------------------------------------------
\subsection{Funcionalidades do Produto}\label{funcionalidades}
Este produto pensado como uma tomada que pode ser adquirida nos Supermercados/Internet. Consiste no rolo que está dentro da parede e pode
ser puxado consoante a necessidade do utilizador, sem ser necessário comprar um extensor. Além disso,
seria possível controlar a mesma por uma interface na mesma ou por aplicação.
Pode ser comprada consoante a necessidade - pequeno, médio, grande, extra grande (1,5,10,20) metros.
As funcionalidades do produto foram definidas após uma reunião do grupo, onde foi analisado que o produto inicial não tinha grande valor
e que poderia ser melhor explorados, sendo que o mesmo foi alterado para ter as seguintes funcionalidades:
\begin{itemize}
	\item Controlo remoto de dispositivos eletrónicos através de um aplicativo móvel.
	\item Conexão com smartphones e tablets via Wi-Fi e/ou Bluetooth.
	\item Monitorização do consumo energético dos dispositivos conectados.
	\item Agendamento de horários para ligar/desligar os dispositivos.
	\item Integração com assistentes virtuais (ex: Google Assistant, Amazon Alexa).
	\item Notificações em tempo real sobre o estado dos dispositivos conectados.
	\item Possibilidade de criar cenários personalizados (ex: "modo férias", "modo noite").
	\item Estender ou enrolar o cabo da tomada de acordo com a necessidade do utilizador.
	      \begin{itemize}
		      \item Manualmente: O utilizador pode estender ou enrolar o cabo da tomada manualmente, de acordo com a sua necessidade.
		      \item Automaticamente: O utilizador pode utilizar o aplicativo para estender ou enrolar o cabo da tomada automaticamente.
	      \end{itemize}
\end{itemize}
%---------------------------------------------------------------------------------------------------------------------------
\subsubsection{Vantagens do Produto}\label{vantagens}
As vantagens deste produto são relacionadas com as suas funcionalidades que fazem com que a vida do utilizador seja mais fácil e prática,
sem necessidade de comprar uma extensão, uma tomada inteligente entre outras coisas, já que este produto agrega tudo isso.
Além disso, o produto é facilmente adaptável a qualquer casa ou espaço, e pode ser adquirido consoante a necessidade do utilizador,
sem necessidade de comprar um produto que seja maior ou menor do que o necessário.
%---------------------------------------------------------------------------------------------------------------------------
\subsubsection{Desvantagens do Produto}\label{desvantagens}
As desvantagens deste produto são relacionadas com o seu custo, que pode ser elevado para alguns utilizadores devido à existência de diversas
funcionalidades e ser algo novo no mercado. Sem considerar o custo, a maior desvantagem é a sua complexidade, já que o utilizador tem de ter algum
cuidado com o produto, já que o mesmo pode ser danificado se não for utilizado da forma correta.

%---------------------------------------------------------------------------------------------------------------------------
\subsection{Mercado do Produto}\label{mercado}
Nesta parte será abordada a análise do mercado, onde será analisado o público alvo, dimensão e potencial de crescimento do
produto, e por fim,a concorrência atual no mercado.

%---------------------------------------------------------------------------------------------------------------------------

\subsubsection{Público Alvo}\label{publico}
O público alvo deste produto é bastante abrangente, já que o objetivo é que o mesmo seja utilizado por qualquer pessoa, independentemente da sua idade,
e que facilite a vida da mesma. O objetivo é que os utilizadores não precisem de comprar imensos produtos para ter o mesmo resultado de um único produto.
%--------------------------------------MARTINHO TRABALHO STARTS HERE----------------------------------------------------------------------

\subsubsection{Dimensão e Potencial do Crescimento}\label{preco}
O produto tem um grande potencial de vendas dado que utilização de espaços de forma eficiente e o uso de produtos
"Smart Home" tem cada vez a crescer mais, então com a utilização inteligente de anuncios será possivel aumentar facilmente
a percentagem de dominancia no mercado.
%------------------------------------MARTINHO TRABALHO ENDS HERE----------------------------------------------------------

\subsubsection{Possíveis Concorrentes}\label{concorrentes}
O produto tem como concorrentes as tomadas inteligentes que existem atualmente no mercado, que são bastante conhecidas e
utilizadas pelos utilizadores, como a \textit{TP-Link} (\cite{tplink}) e a \textit{Xiaomi} (\cite{xiaomi}). Porém, são apenas
tomadas inteligentes, que não tem a funcionalidade de estender ou enrolar o cabo, obrigando assim o utilizador a comprar um extensor,
se precisar de mais metros de cabo.
%-----------------------------------PAULO TRABALHO STARTS HERE----------------------------------------------------------------

\subsection{Meio Envolvente}\label{meio}
Nesta parte será abordado o meio envolvente do produto, onde serão analisadas as ameaças e oportunidades que o mesmo tem, bem
como os cenários futuros e tendências que podem existir para o produto.
%---------------------------------------------------------------------------------------------------------------------------

\subsubsection{Ameaças e Oportunidades}\label{ameacas}
As oportunidades que este produto possui estão relacionadas com o aumento da procura por produtos que facilitem a vida dos utilizadores, 
e que sejam capazes de automatizar a residência. Além disso, a praticidade e a facilidade de utilização do produto são
também uma grande oportunidade, já que o mesmo pode ser utilizado por qualquer pessoa, independentemente da sua idade ou conhecimentos técnicos. 
Outra oportunidade é a capacidade de personalização ao nível do utilizador, já que o mesmo pode escolher o tamanho do cabo que pretende,
o que faz com que o produto seja bastante adaptável a qualquer espaço e ainda pode criar rotinas personalizadas para o mesmo.

As ameaças que serão enfrentadas por este produto são relacionadas com a concorrência, já que existem atualmente no mercado, e 
uma nova empresa com um produto que nunca foi visto antes, pode ser considerado um investimento arriscado, pois não se sabe se 
é uma empresa ou um produto com qualidade e fiabilidade. Logo, muitos utilizadores podem preferir comprar um produto de uma marca 
que já possui uma imagem de marca confiável.
%---------------------------------------------------------------------------------------------------------------------------

\subsubsection{Cenários Futuros e Tendências}\label{cenarios}
Existem várias possibilidades do que pode acontecer no futuro, considerando que este produto irá ser um sucesso, e que irá ser 
adquirido por muitos utilizadores. Uma das possibilidades é que o produto seja modificado para ter mais versões, como por exemplo, 
uma versão que tenha mais do que uma tomada e que seja possível controlar cada uma delas individualmente. Outra possibilidade é que 
o produto seja sempre atualizado para ser mais sustentável a nível energético e ambiental, seja na parte de software ou hardware. 
É desejável ainda que o produto seja compatível com o software de outras marcas, como por exemplo, a \textit{SmartThings} (\cite{smartthings}) 
da Samsung ou a \textit{Google Home} (\cite{googlehome}), para que o utilizador possa controlar o produto através de um único aplicativo,
sem necessidade de ter vários aplicativos para controlar os seus dispositivos.
%------------------------------PAULO TRABALHO ENDS HERE-------------------------------------------------------------------------
%------------------------------SOUSA TRABALHO START--------------------------------------------------------------------------------
\subsection{Exequibilidade do Marketing}\label{exequibilidade}
%---------------------------------------------------------------------------------------------------------------------------

\subsubsection{Posicionamento}\label{posicionamento}
%---------------------------------------------------------------------------------------------------------------------------
\subsubsection{Mix de Produto}\label{mixprod}
%---------------------------------------------------------------------------------------------------------------------------
\subsubsection{Mix de Preço}\label{mixpreco}
%---------------------------------------------------------------------------------------------------------------------------
\subsubsection{Mix de Canais de Distribuição}\label{mixcanais}
%---------------------------------------------------------------------------------------------------------------------------
\subsubsection{Mix de Comunicação}\label{mixcomunicacao}
%---------------------------------------------------------------------------------------------------------------------------
\subsubsection{Previsão de Vendas}\label{previsao}

%----------------------------SOUSA TRABALHO ENDS HERE----------------------------------------------------
%----------------------------MARTINHO TRABALHO STARTS HERE---------------------------------------------------------------
\subsection{Exequibilidade das Operações}\label{exequibilidadeop}
Estes equipamentos (Extensões inteligentes e caixas com rolo) são de fácil produção permitindo que os custos se mantenham baixos,
não só cada um dos equipamentos já é vendido separadamente durante várias gerações, diminuindo os custos de produção, como também
a produção em massa de cada um deles, permitindo que o produto final seja vendido a um preço acessível ao consumidor final.
%---------------------------------------------------------------------------------------------------------------------------
\subsubsection{Processo e Capacidade}\label{processo}
Como já referido o processo de produção é relativamente simples, isso significa que a produção em alto volume para satisfazer a procura é exquivel,
não só como é um produto novo de uma marca nova a procura não irá ser tão elevada.
%---------------------------------------------------------------------------------------------------------------------------
\subsubsection{Recursos Humanos}\label{recursos}
A nossa empresa irá constituir os melhores funcionarios qualificados que podemos arranjar, iremos ter em consideração a satisfação e qualquer
input vindo dos nossos funcionarios de modo a constantemente evoluir o produto e ter as melhores conduções de trabalho no mercado, 
esta tática de aperfeiçoamento constante é chamada de "Kaizen".
%---------------------------------------------------------------------------------------------------------------------------
\subsubsection{Localização das Instalações}\label{localizacao}
As instalações principais irão estar localizadas em Portugal de modo a garantir mais postos de trabalho no pais, para facilitar
em termos de portes de envio e tempo de entrega irão estar distribuidos armazens em locais estratégicos como Estados Unidos,
Taiwan e Brasil.

%----------------------------MARTINHO TRABALHO ENDS HERE---------------------------------------------------------------
%----------------------------PAULO TRABALHO STARTS HERE---------------------------------------------------------------

\subsection{Impacto Socioeconómico}\label{impacto}
%---------------------------------------------------------------------------------------------------------------------------
\subsubsection{Emprego Qualificado}\label{emprego}
%---------------------------------------------------------------------------------------------------------------------------

\subsubsection{Parcerias Tecnologicas}\label{parcerias}
%---------------------------------------------------------------------------------------------------------------------------
\subsubsection{Sinergias com Outras Atividades}\label{sinergias}
%---------------------------------------------------------------------------------------------------------------------------
\subsubsection{Potencial de Crescimento}\label{crescimento}


%-----------------------------------PAULO TRABALHO ENDS HERE-------------------------------------------------------------------

\newpage

%---------------------------------------------------------------------------------------------------------------------------
\section{Conclusão}\label{con}

Concluimos que este buraco no mercado irá permitir que o produto tenha um grande sucesso, dado ao simples facto de que a junção destas
tecnologias nunca foi realizada antes. O produto é inovador e irá permitir que o utilizador tenha um maior controlo sobre os seus dispositivos eletrónicos,
sem ser necessário comprar um extensor. Além disso, o produto é fácil de utilizar e pode ser adquirido em qualquer supermercado ou na internet,
abrindo as portas a um grande mercado.
%---------------------------------------------------------------------------------------------------------------------------

\newpage
\renewcommand{\refname}{Bibliografia} % Para artigos
\renewcommand{\bibname}{Bibliografia} % Para livros e relatórios
\addcontentsline{toc}{section}{Bibliografia} % Adiciona a Bibliografia ao índice
\printbibliography

\end{document}
