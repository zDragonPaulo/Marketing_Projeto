\documentclass[a4paper]{article}
% Pacotes necessários
\usepackage[portuguese]{babel}
\usepackage[backend=biber, style=apa, citestyle=apa, language=portuguese]{biblatex}
\usepackage{csquotes}
\addbibresource{Recursos/referencias.bib}

\usepackage{amsmath}
\usepackage{graphicx}
\usepackage{subcaption}
\usepackage{setspace}
\usepackage{siunitx} % Required for alignment
\sisetup{
  round-mode          = places, % Rounds numbers
  round-precision     = 2, % to 2 places
}
\usepackage{enumerate}
\usepackage{enumitem}
\usepackage{amsmath}
\usepackage{karnaugh-map}
\usepackage[section]{placeins}
\usepackage{geometry}
\usepackage{amssymb}
\usepackage{titling}
\usepackage[T1]{fontenc}
\usepackage{float}
\usepackage[hidelinks]{hyperref}
\usepackage{xcolor}
\usepackage{indentfirst}
\usepackage{array}
\usepackage{soul}
\usepackage{afterpage}
\newcolumntype{P}[1]{>{\centering\arraybackslash}p{#1}}
\onehalfspacing


% Comando para criar uma página vazia
\newcommand\myemptypage{
    \null
    \thispagestyle{empty}
    \addtocounter{page}{-1}
    \newpage
}

% Página de título principal
\newcommand{\firsttitlepage}{
    \begin{titlepage}
        \centering
        \vspace*{1cm}
        
        % Logos superior
        \begin{figure}[h!]
            \centering
            \includegraphics[width=6cm]{Recursos/LOGO_IPB} % Substitua pelo caminho da imagem
            \vspace{0.5cm}
        \end{figure}

        % Informações da instituição
        \large\textbf{INSTITUTO POLITÉCNICO DE BEJA} \\
        \large\textbf{Escola Superior de Tecnologia e Gestão} \\
        \large\textbf{Licenciatura em Engenharia Informática} \\
        \large\textbf{Regulação Informática} \\
        
        \vspace{2cm}
        
        % Título do projeto
        {\Huge \textbf{Caso 1}} \\
        
        \vspace{1.5cm}
        
        % Autores
        \large Paulo António Tavares Abade - 23919 \\
        
        \vfill
        
        % Logo inferior
        \begin{figure}[h!]
            \centering
            \includegraphics[width=6cm]{Recursos/IPBejaESTIG.jpg} % Substitua pelo caminho da imagem
        \end{figure}
        
        \vspace{1cm}
        
        % Local e data
        {\large Beja, abril de 2025}
    \end{titlepage}
}

\newcommand{\secondtitlepage}{
    \begin{titlepage}
        \centering
        \vspace*{1cm}
        
        % Informações da instituição
        \large\textbf{INSTITUTO POLITÉCNICO DE BEJA} \\
        \large\textbf{Escola Superior de Tecnologia e Gestão} \\
        \large\textbf{Licenciatura em Engenharia Informática} \\
        \large\textbf{Regulação Informática} \\
        
        \vspace{2cm}
        
        % Título do projeto
        {\Huge \textbf{Caso 1}} \\
        
        \vspace{1.5cm}
        
        % Autores
        \large Paulo António Tavares Abade - 23919 \\

        \vspace{2cm}

        % Orientador
        \large Orientador: Manuel David Rodrigues Masseno \\
        
        \vfill
        
        % Local e data
        {\large Beja, abril de 2025}
    \end{titlepage}
}

\begin{document}


\pagenumbering{gobble} % Oculta numeração da página

% Primeira página de título
\firsttitlepage

\secondtitlepage


% Abstract
\section*{\LARGE\textbf{\textit{Resumo}}}

Resposta ao Caso 1 da disciplina de Regulação Informática, onde foi necessário analisar informação...


\vspace{1cm}
% Keywords
\textbf{Keywords:} rgdp
\newpage
%--------------------------------------------------------------------------------------------------------------------------------------

\section*{\LARGE\textbf{\textit{Abstract}}}

Response to Case 1 of the discipline of Computer Regulation, where it was necessary to analyze information...




\vspace{1cm}
% Keywords
\textbf{Keywords:} rgdp
\renewcommand{\contentsname}{Índice}       % Título do sumário
\renewcommand{\listfigurename}{Índice de Figuras} % Título da lista de figuras

% Início do conteúdo do relatório
\newpage
\doublespacing
\tableofcontents
\listoffigures
\doublespacing

\newpage
\pagenumbering{arabic}

\section{Introdução}\label{intro}
Alguma coisa sobre o trabalho, o que foi feito, o que foi analisado, etc.
%---------------------------------------------------------------------------------------------------------------------------
------
\section{Desenvolvimento}\label{etl}
O trabalho de ETL foi realizado em 3 fases, a primeira foi a extração dos dados, 
a segunda a transformação e por fim a inserção dos dados em tabelas dinâmicas para 
facilitar a visualização dos dados ao visualizador.
\newpage

%---------------------------------------------------------------------------------------------------------------------------
\section{Conclusão}\label{con}
Estas Crises Petrolíferas fizeram com que os fabricantes de automóveis tivessem de se adaptar e produzir carros mais eficientes, estas adaptações para 
veiculos americanos foram dificeis, cujos quais eram conhecidos por serem grandes e terem motores grandes que gastam muito combustível,
tiveram a sua potencia drasticamente diminuida ao ponto dos modelos desportivos serem considerados muito lentos para a sua categoria.
Os japoneses aproveitaram a situação pois já produziam carros mais pequenos e eficientes e decidiram começar a vender mais no mercado americano,
aumentando a sua quota de mercado significativamente.
%---------------------------------------------------------------------------------------------------------------------------

\newpage
\renewcommand{\refname}{Bibliografia} % Para artigos
\renewcommand{\bibname}{Bibliografia} % Para livros e relatórios
\addcontentsline{toc}{section}{Bibliografia} % Adiciona a Bibliografia ao índice
\printbibliography

\end{document}
