\documentclass[a4paper]{article}
% Pacotes necessários
\usepackage[portuguese]{babel}
\usepackage[backend=biber, style=apa, citestyle=apa, language=portuguese]{biblatex}
\usepackage{csquotes}
\addbibresource{Recursos/referencias.bib}

\usepackage{amsmath}
\usepackage{graphicx}
\usepackage{subcaption}
\usepackage{setspace}
\usepackage{siunitx} % Required for alignment
\sisetup{
  round-mode          = places, % Rounds numbers
  round-precision     = 2, % to 2 places
}
\usepackage{enumerate}
\usepackage{enumitem}
\usepackage{amsmath}
\usepackage{karnaugh-map}
\usepackage[section]{placeins}
\usepackage{geometry}
\usepackage{amssymb}
\usepackage{titling}
\usepackage[T1]{fontenc}
\usepackage{float}
\usepackage[hidelinks]{hyperref}
\usepackage{xcolor}
\usepackage{indentfirst}
\usepackage{array}
\usepackage{soul}
\usepackage{afterpage}
\newcolumntype{P}[1]{>{\centering\arraybackslash}p{#1}}
\onehalfspacing


% Comando para criar uma página vazia
\newcommand\myemptypage{
    \null
    \thispagestyle{empty}
    \addtocounter{page}{-1}
    \newpage
}

% Página de título principal
\newcommand{\firsttitlepage}{
    \begin{titlepage}
        \centering
        \vspace*{1cm}
        
        % Logos superior
        \begin{figure}[h!]
            \centering
            \includegraphics[width=6cm]{Recursos/LOGO_IPB} % Substitua pelo caminho da imagem
            \vspace{0.5cm}
        \end{figure}

        % Informações da instituição
        \large\textbf{INSTITUTO POLITÉCNICO DE BEJA} \\
        \large\textbf{Escola Superior de Tecnologia e Gestão} \\
        \large\textbf{Licenciatura em Engenharia Informática} \\
        \large\textbf{Marketing e Empreendorismo} \\
        
        \vspace{1.0cm}
        
        % Título do projeto
        {\Huge \textbf{EzPower™}} \\
        
        \vspace{1.5cm}
        
        % Autores
        \large João Eduardo Sara Sousa - 23916 \\
        \large Martinho José Novo Caeiro - 23917 \\
        \large Paulo António Tavares Abade - 23919 \\

        \vfill
        
        % Logo inferior
        \begin{figure}[h!]
            \centering
            \includegraphics[width=6cm]{Recursos/IPBejaESTIG.jpg} % Substitua pelo caminho da imagem
        \end{figure}
        
        
        % Local e data
        {\large Beja, abril de 2025}
    \end{titlepage}
}

\newcommand{\secondtitlepage}{
    \begin{titlepage}
        \centering
        \vspace*{1cm}
        
        % Informações da instituição
        \large\textbf{INSTITUTO POLITÉCNICO DE BEJA} \\
        \large\textbf{Escola Superior de Tecnologia e Gestão} \\
        \large\textbf{Licenciatura em Engenharia Informática} \\
        \large\textbf{Marketing e Empreendorismo} \\
        
        \vspace{2cm}
        
        % Título do projeto
        {\Huge \textbf{EzPower™}} \\
        
        \vspace{1.5cm}
        
        % Autores
        \large João Eduardo Sara Sousa - 23916 \\
        \large Martinho José Novo Caeiro - 23917 \\
        \large Paulo António Tavares Abade - 23919 \\

        \vspace{2cm}

        % Orientador
        \large Orientador: Eunice Duarte\\
        
        \vfill
        
        % Local e data
        {\large Beja, abril de 2025}
    \end{titlepage}
}

\begin{document}


\pagenumbering{gobble} % Oculta numeração da página

% Primeira página de título
\firsttitlepage

\secondtitlepage


% Abstract
\section*{\LARGE\textbf{\textit{Resumo}}}

Um produto imperdível para quem procura uma solução de qualidade e eficiência.


\vspace{1cm}
% Keywords
\textbf{Keywords:} rgdp
\newpage
%--------------------------------------------------------------------------------------------------------------------------------------

\section*{\LARGE\textbf{\textit{Abstract}}}

A must-have product for those looking for a quality and efficient solution.



\vspace{1cm}
% Keywords
\textbf{Keywords:} rgdp
\renewcommand{\contentsname}{Índice}       % Título do sumário
\renewcommand{\listfigurename}{Índice de Figuras} % Título da lista de figuras

% Início do conteúdo do relatório
\newpage
\doublespacing
\tableofcontents
\listoffigures
\doublespacing

\newpage
\pagenumbering{arabic}

\section{Introdução}\label{intro}
Neste projeto irá ser abordado como foi o desenvolvimento deste produto e como foram separadas as tarefas para cada membro do grupo.
O produto em questão é uma tomada inteligente que tem como objetivo facilitar a vida dos utilizadores, permitindo o controlo
remoto de dispositivos eletrónicos através de um aplicativo móvel. A tomada é equipada com tecnologia Wi-Fi e Bluetooth,
permitindo a conexão com smartphones e tablets.
%---------------------------------------------------------------------------------------------------------------------------
\section{Desenvolvimento}\label{etl}
Nesta parte do relatório será abordado o desenvolvimento do projeto, desde a sua preparação, desenho e forma final. Será também
abordada a separação de tarefas entre os membros do grupo, bem como a forma como cada um contribuiu para o projeto.
%---------------------------------------------------------------------------------------------------------------------------

\subsection{Metodologia de Trabalho}\label{metodologia}
A metodologia de trabalho utilizada foi uma variação do SCRUM (\cite{SCRUM}), onde cada membro do grupo tinha tarefas específicas a realizar.
Foram realizadas reuniões pela plataforma \textit{Discord} (\cite{discord}) eram feitas com variações de 1 a 2 dias de intervalo,
onde cada membro do grupo apresentava o que tinha feito e o  que iria fazer a seguir. A confirmação da finalização das tarefas
era feita através de e-mails, onde cada membro do grupo  utilizava o seu e-mail institucional para informar os restantes
membros do grupo e o professor orientador.
%---------------------------------------------------------------------------------------------------------------------------
\subsection{Separação de Tarefas}\label{separacao}
A separação de tarefas foi feita de forma a que cada membro do grupo tivesse tarefas específicas a realizar, sendo que ficou dividido
da seguinte forma:
\begin{itemize}
	\item João Eduardo Sara Sousa - 23916: Exequibilidade de Marketing e Powerpoint.
	\item Martinho José Novo Caeiro - 23917: Exequibilidade do Produto e Excel.
	\item Paulo António Tavares Abade - 23919: Restante relatório e Powerpoint.
\end{itemize}
%---------------------------------------------------------------------------------------------------------------------------
\subsection{Desenho do Produto}\label{desenho}
O desenho do produto foi feito pelo membro do grupo Martinho José Novo Caeiro - 23917, sendo que o mesmo utilizou um desenho
feito à mão e utilizou o \textit{ChatGPT} (\cite{chatgpt}) para o ajudar a criar o desenho do produto.
O desenho foi feito com o objetivo de ser o mais simples possível, de forma a que o utilizador consiga perceber como funciona o produto e como
que funcionalidade é que teria. Após uma conversa do grupo, o desenho foi alterado para que fosse ainda mais inovador e agregasse mais valor ao
produto, adicionando tecnologias que não estavam previstas inicialmente como o \textit{Bluetooth} e o \textit{Wi-Fi}.
%---------------------------------------------------------------------------------------------------------------------------
\subsection{Funcionalidades do Produto}\label{funcionalidades}
Este produto pensado como uma tomada que pode ser adquirida nos Supermercados/Internet. Consiste no rolo que está dentro da parede e pode
ser puxado consoante a necessidade do utilizador, sem ser necessário comprar um extensor. Além disso,
seria possível controlar a mesma por uma interface na mesma ou por aplicação.
Pode ser comprada consoante a necessidade - pequeno, médio, grande, extra grande (1,5,10,20) metros.
As funcionalidades do produto foram definidas após uma reunião do grupo, onde foi analisado que o produto inicial não tinha grande valor
e que poderia ser melhor explorados, sendo que o mesmo foi alterado para ter as seguintes funcionalidades:
\begin{itemize}
	\item Controlo remoto de dispositivos eletrónicos através de um aplicativo móvel.
	\item Conexão com smartphones e tablets via Wi-Fi e/ou Bluetooth.
	\item Monitorização do consumo energético dos dispositivos conectados.
	\item Agendamento de horários para ligar/desligar os dispositivos.
	\item Integração com assistentes virtuais (ex: Google Assistant, Amazon Alexa).
	\item Notificações em tempo real sobre o estado dos dispositivos conectados.
	\item Possibilidade de criar cenários personalizados (ex: "modo férias", "modo noite").
	\item Estender ou enrolar o cabo da tomada de acordo com a necessidade do utilizador.
	      \begin{itemize}
		      \item Manualmente: O utilizador pode estender ou enrolar o cabo da tomada manualmente, de acordo com a sua necessidade.
		      \item Automaticamente: O utilizador pode utilizar o aplicativo para estender ou enrolar o cabo da tomada automaticamente.
	      \end{itemize}
\end{itemize}
%---------------------------------------------------------------------------------------------------------------------------
\subsubsection{Vantagens do Produto}\label{vantagens}
As vantagens deste produto são relacionadas com as suas funcionalidades que fazem com que a vida do utilizador seja mais fácil e prática,
sem necessidade de comprar uma extensão, uma tomada inteligente entre outras coisas, já que este produto agrega tudo isso.
Além disso, o produto é facilmente adaptável a qualquer casa ou espaço, e pode ser adquirido consoante a necessidade do utilizador,
sem necessidade de comprar um produto que seja maior ou menor do que o necessário.
%---------------------------------------------------------------------------------------------------------------------------
\subsubsection{Desvantagens do Produto}\label{desvantagens}
As desvantagens deste produto são relacionadas com o seu custo, que pode ser elevado para alguns utilizadores devido à existência de diversas
funcionalidades e ser algo novo no mercado. Sem considerar o custo, a maior desvantagem é a sua complexidade, já que o utilizador tem de ter algum
cuidado com o produto, já que o mesmo pode ser danificado se não for utilizado da forma correta.

%---------------------------------------------------------------------------------------------------------------------------
\subsection{Mercado do Produto}\label{mercado}
Nesta parte será abordada a análise do mercado, onde será analisado o público alvo, dimensão e potencial de crescimento do
produto, e por fim,a concorrência atual no mercado.

%---------------------------------------------------------------------------------------------------------------------------

\subsubsection{Público Alvo}\label{publico}
O público alvo deste produto é bastante abrangente, já que o objetivo é que o mesmo seja utilizado por qualquer pessoa, independentemente da sua idade,
e que facilite a vida da mesma. O objetivo é que os utilizadores não precisem de comprar imensos produtos para ter o mesmo resultado de um único produto.
%--------------------------------------MARTINHO TRABALHO STARTS HERE----------------------------------------------------------------------

\subsubsection{Dimensão e Potencial do Crescimento}\label{preco}
O produto tem um grande potencial de vendas dado que utilização de espaços de forma eficiente e o uso de produtos
"Smart Home" tem cada vez a crescer mais, então com a utilização inteligente de anuncios será possivel aumentar facilmente
a percentagem de dominancia no mercado.
%------------------------------------MARTINHO TRABALHO ENDS HERE----------------------------------------------------------

\subsubsection{Possíveis Concorrentes}\label{concorrentes}
O produto tem como concorrentes as tomadas inteligentes que existem atualmente no mercado, que são bastante conhecidas e
utilizadas pelos utilizadores, como a \textit{TP-Link} (\cite{tplink}) e a \textit{Xiaomi} (\cite{xiaomi}). Porém, são apenas
tomadas inteligentes, que não tem a funcionalidade de estender ou enrolar o cabo, obrigando assim o utilizador a comprar um extensor,
se precisar de mais metros de cabo.
%-----------------------------------PAULO TRABALHO STARTS HERE----------------------------------------------------------------

\subsection{Meio Envolvente}\label{meio}
Nesta parte será abordado o meio envolvente do produto, onde serão analisadas as ameaças e oportunidades que o mesmo tem, bem
como os cenários futuros e tendências que podem existir para o produto.
%---------------------------------------------------------------------------------------------------------------------------

\subsubsection{Ameaças e Oportunidades}\label{ameacas}
As oportunidades que este produto possui estão relacionadas com o aumento da procura por produtos que facilitem a vida dos utilizadores,
e que sejam capazes de automatizar a residência. Além disso, a praticidade e a facilidade de utilização do produto são
também uma grande oportunidade, já que o mesmo pode ser utilizado por qualquer pessoa, independentemente da sua idade ou conhecimentos técnicos.
Outra oportunidade é a capacidade de personalização ao nível do utilizador, já que o mesmo pode escolher o tamanho do cabo que pretende,
o que faz com que o produto seja bastante adaptável a qualquer espaço e ainda pode criar rotinas personalizadas para o mesmo.

As ameaças que serão enfrentadas por este produto são relacionadas com a concorrência, já que existem atualmente no mercado, e
uma nova empresa com um produto que nunca foi visto antes, pode ser considerado um investimento arriscado, pois não se sabe se
é uma empresa ou um produto com qualidade e fiabilidade. Logo, muitos utilizadores podem preferir comprar um produto de uma marca
que já possui uma imagem de marca confiável.
%---------------------------------------------------------------------------------------------------------------------------

\subsubsection{Cenários Futuros e Tendências}\label{cenarios}
Existem várias possibilidades do que pode acontecer no futuro, considerando que este produto irá ser um sucesso, e que irá ser
adquirido por muitos utilizadores. Uma das possibilidades é que o produto seja modificado para ter mais versões, como por exemplo,
uma versão que tenha mais do que uma tomada e que seja possível controlar cada uma delas individualmente. Outra possibilidade é que
o produto seja sempre atualizado para ser mais sustentável a nível energético e ambiental, seja na parte de software ou hardware.
É desejável ainda que o produto seja compatível com o software de outras marcas, como por exemplo, a \textit{SmartThings} (\cite{smartthings})
da Samsung ou a \textit{Google Home} (\cite{googlehome}), para que o utilizador possa controlar o produto através de um único aplicativo,
sem necessidade de ter vários aplicativos para controlar os seus dispositivos.
%------------------------------PAULO TRABALHO ENDS HERE-------------------------------------------------------------------------
%------------------------------SOUSA TRABALHO START--------------------------------------------------------------------------------
\subsection{Exequibilidade do Marketing}\label{exequibilidade}
Nesta parte será abordada a exequibilidade do produto, analisando a viabilidade da sua entrada e crescimento no mercado através de estratégias de marketing bem definidas. Serão explorados fatores como o posicionamento do produto, a construção da proposta de valor, os canais de distribuição a utilizar, as estratégias de comunicação e a previsão de vendas.
%---------------------------------------------------------------------------------------------------------------------------
\subsubsection{Posicionamento}\label{posicionamento}
Este produto será colocado como uma solução moderna, sendo acessível e facíl de se usar por qualquer pessoa que deseja automatizar a sua casa.
O produto foi desenvolvido com o objetivo de democratizar o acesso à tecnologia smart home, permitindo que utilizadores com diferentes níveis de conhecimento técnico consigam tirar partido das suas funcionalidades avançadas sem dificuldades.
%---------------------------------------------------------------------------------------------------------------------------
\subsubsection{Mix de Produto}\label{mixprod}
As características técnicas e funcionais do produto já foram apresentadas detalhadamente nas secções 2.4 e 2.4.1 anteriormente. 
Neste ponto, irá falar-se de como este produto é uma solução multifuncional, pois combina as características de uma extensão, uma tomada inteligente e dispositivo de automatização doméstica.

A embalagem terá um papel importante na experiência inicial do consumidor, com um visual apelativo, materiais sustentáveis e a inclusão de um código QR para fácil acesso à aplicação móvel.
%---------------------------------------------------------------------------------------------------------------------------
\subsubsection{Mix de Preço}\label{mixpreco}
O preço do produto foi definido de maneira a que exista um equilíbrio entre custo e acessibilidade. Tendo esta uma estratégia para dois mercados:
\begin{itemize}
	\item 49,99€ para o mercado nacional (Portugal).
	\item 54,99€ para o mercado internacional.
\end{itemize}
Esta diferença existe pois, há diferença no custo de exportação para o exterior. Para facilitar a entrada no mercado, estão previstas ações promocionais de lançamento, como descontos temporários para os primeiros clientes, ofertas em packs (ex: duas unidades com preço reduzido) e campanhas com portes gratuitos em compras superiores a um determinado valor.

%---------------------------------------------------------------------------------------------------------------------------
\newpage
\subsubsection{Mix de Canais de Distribuição}\label{mixcanais}
A distribuição do produto será feita através de canais diversificados, combinando estratégias online e físicas para garantir uma presença sólida e acessível ao consumidor.

No canal digital, o produto será vendido:
\begin{itemize}
    \item Através de uma loja online própria, com interface intuitiva e entregas rápidas.
    \item Em plataformas de e-commerce populares como Amazon, Fnac, Worten, entre outras.
\end{itemize}
Paralelamente, será implementada uma estratégia de distribuição em lojas físicas de grande retalho, como o Continente, bem como outras superfícies comerciais com elevada afluência de consumidores.
Esta abordagem visa tornar o produto mais visível e acessível ao público em geral, aproveitando o tráfego constante destes espaços.
%---------------------------------------------------------------------------------------------------------------------------
\subsubsection{Mix de Comunicação}\label{mixcomunicacao}
A comunicação do produto será centrada numa abordagem digital, acessível e próxima do consumidor, com o objetivo de destacar a sua inovação, facilidade de uso e utilidade no dia a dia.

A estratégia passa por utilizar canais de comunicação direta e indireta, incluindo:
\begin{itemize}
    \item Redes sociais como Instagram, Facebook e TikTok, com conteúdos curtos, visuais e explicativos (vídeos demonstrativos, tutoriais e casos de uso prático).
    \item Campanhas de publicidade online (Google Ads, Meta Ads) direcionadas ao público-alvo identificado.
    \item Parcerias com construtoras, decoradores e empresas de renovação de imóveis, promovendo o produto como parte de kits ou soluções integradas para casas inteligentes.
\end{itemize}

A mensagem principal será centrada na conveniência, controlo e modernidade, destacando o produto como uma solução prática para tornar qualquer casa mais inteligente.
%---------------------------------------------------------------------------------------------------------------------------
\newpage
\subsubsection{Previsão de Vendas}\label{previsao}
A previsão de vendas do produto foi desenvolvida com base em cenários progressivos de crescimento, considerando a entrada gradual no mercado, o alargamento dos canais de distribuição e o impacto esperado das ações de comunicação, tanto para o público em geral como para profissionais da construção e remodelação.
O planeamento comercial teve em conta:
\begin{itemize}
    \item A penetração inicial no mercado nacional, através de canais online e pontos de venda físicos como retalhistas de grande consumo (ex: Continente);
    \item A posterior expansão para mercados internacionais, suportada por armazéns logísticos estratégicos e presença em marketplaces globais;
    \item O potencial de vendas em volume através de parcerias com house builders, arquitetos e promotores imobiliários;
    \item A possibilidade de vendas por packs e campanhas promocionais em períodos-chave (lançamento, Natal, Black Friday, etc.).
\end{itemize}
As projeções detalhadas foram organizadas num ficheiro Excel, que acompanha este relatório.

%----------------------------SOUSA TRABALHO ENDS HERE----------------------------------------------------
%----------------------------MARTINHO TRABALHO STARTS HERE---------------------------------------------------------------
\newpage
\subsection{Exequibilidade das Operações}\label{exequibilidadeop}
A produção dos equipamentos — como as extensões inteligentes e as caixas com rolo — apresentam elevada viabilidade, tanto técnica quanto económica.
Esses dispositivos, já comercializados separadamente em diferentes versões ao longo do tempo, possuem processos de fabrico consolidados e otimizados,
o que reduz significativamente os custos. A produção em larga escala desses componentes possibilita um preço final competitivo, tornando o produto acessível a um público mais amplo.
%---------------------------------------------------------------------------------------------------------------------------
\subsubsection{Processo e Capacidade}\label{processo}
O processo de produção caracteriza-se pela sua simplicidade e escalabilidade. Tal característica permite aumentar o volume de fabrico rapidamente,
ajustando-se à procura do mercado. Além disso, por se tratar de um produto inovador de uma marca emergente, espera-se uma procura inicial moderada,
o que facilita a gestão dos recursos e dos prazos de entrega nos primeiros ciclos de comercialização.
%---------------------------------------------------------------------------------------------------------------------------
\subsubsection{Recursos Humanos}\label{recursos}
A empresa pretende formar uma equipa composta por profissionais altamente qualificados e motivados. A valorização do contributo individual,
aliada a um ambiente de trabalho positivo e participativo, será uma prioridade. A estratégia de melhoria contínua adotada, baseada na filosofia japonesa "Kaizen" (\cite{kaizen}) ,
cuja qual visa o aperfeiçoamento constante dos processos, produtos e condições laborais, promovendo inovação e eficiência.
%---------------------------------------------------------------------------------------------------------------------------
\subsubsection{Localização das Instalações}\label{localizacao}
A sede e a principal unidade de produção estarão localizadas em Portugal, contribuindo para o desenvolvimento económico local e para a criação de postos de trabalho no país.
Complementarmente, serão estabelecidos armazéns logísticos em pontos estratégicos — como Estados Unidos, Taiwan e Brasil — com o objetivo de reduzir os prazos de entrega,
otimizar os custos de distribuição e facilitar o acesso ao mercado internacional.

%---------------------------------------------------------------------------------------------------------------------------
\subsubsection{Fases de Montagem}\label{fasesmontagem}

O processo de montagem dos equipamentos será estruturado em fases bem definidas, com o objetivo de garantir eficiência, padronização e qualidade em todas as unidades produzidas. As principais etapas são:

\begin{itemize} \item \textbf{1. Preparação dos Componentes:} Receção e verificação de qualidade de todos os componentes essenciais, como circuitos eletrónicos, caixas plásticas e mecanismos de rolo. Esta etapa assegura que apenas peças em conformidade seguem para a montagem.

	\item \textbf{2. Montagem Eletrónica:} Integração dos circuitos nas estruturas principais, incluindo a ligação de sensores, cablagem e módulos de controlo. Esta fase exige precisão técnica e será realizada por técnicos especializados ou com apoio de sistemas automatizados.

	\item \textbf{3. Montagem Mecânica:} Instalação dos mecanismos físicos (como os rolos) e encaixe dos componentes estruturais. Aqui também se garante o correto alinhamento e fixação de todas as peças.

	\item \textbf{4. Programação e Configuração:} Inserção do firmware ou software necessário para o funcionamento inteligente do dispositivo. Dependendo do modelo, esta etapa pode incluir calibração de sensores ou conexão com plataformas de gestão.

	\item \textbf{5. Testes de Qualidade:} Cada unidade passa por testes funcionais e de segurança, garantindo que o produto cumpre os padrões definidos e está pronto para utilização.

	\item \textbf{6. Embalamento e Expedição:} Após aprovação nos testes, os produtos são embalados com proteção adequada e preparados para envio aos pontos de venda ou diretamente aos consumidores. \end{itemize}

Este processo modular e escalável permite um controlo de qualidade rigoroso e adaptações rápidas em caso de necessidade de ajustes ou personalizações futuras.
%----------------------------MARTINHO TRABALHO ENDS HERE---------------------------------------------------------------
%----------------------------PAULO TRABALHO STARTS HERE---------------------------------------------------------------

\subsection{Impacto Socioeconómico}\label{impacto}
Nesta parte será abordado o impacto socioeconómico do produto, onde serão analisados os impactos sociais e económicos que o mesmo pode ter.

%---------------------------------------------------------------------------------------------------------------------------
\subsubsection{Emprego Qualificado}\label{emprego}
O desenvolvimento deste produto irá permitir a criação de novos postos de trabalho, já que o mesmo irá ser desenvolvido por uma nova empresa,
e que irá precisar de novos funcionários para o desenvolvimento do mesmo. Dentre os postos de trabalho que irão ser criados, estão os cientistas
que irão desenvolver o produto em si na parte do hardware, onde irão ser criados os componentes elétricos utilizados no produto, e os engenheiros
de software que irão desenvolver o software do produto, onde irão ser criadas as funcionalidades do mesmo. Deve ser considerado ainda que o produto
necessita ser fabricado, desenvolvendo esses esses empregos para fabricá-lo. Depois, ainda serão criados postos de trabalho
relacionados com o marketing e vendas do produto, onde serão criados postos de trabalho relacionados com a venda do produto,
como por exemplo, os vendedores que irão vender o produto em lojas físicas e online, e os responsáveis pelo marketing do produto.
%---------------------------------------------------------------------------------------------------------------------------

\subsubsection{Parcerias Tecnologicas}\label{parcerias}
Como já foi mencionado anteriormente, seria desejável que o produto fosse compatível com o software de outras marcas, como por exemplo,
a \textit{SmartThings} (\cite{smartthings}). Para isso, seria necessário criar parcerias com essas marcas, para que o produto fosse compatível com o software
delas, e que o utilizador pudesse controlar o produto através de um único aplicativo, sem necessidade de ter vários aplicativos para controlar os seus dispositivos.
Um exemplo de um tipo dessa parceria é a \textit{Samsung} e a \textit{WiZ}, onde a WiZ fabrica lâmpadas inteligentes que são compatíveis com o software da Samsung.
%---------------------------------------------------------------------------------------------------------------------------
\subsubsection{Sinergias com Outras Atividades}\label{sinergias}
Este produto pode beneficiar de sinergias com outras atividades, como por exemplo, a sua integração em projetos de construção
de casas inteligentes, onde o produto pode estar presente desde a origem da habitação, aumentando assim a sua visibilidade e impulsionando as vendas.

Outra possibilidade seria a comercialização do produto em \textit{kits} de produtos para casas inteligentes, onde esta tomada faria parte de um conjunto
de dispositivos fiáveis e de qualidade. Essa integração transmite maior confiança ao utilizador, uma vez que o produto é apresentado como parte de um
ecossistema coeso, reforçando assim a imagem de marca e a perceção de valor do produto.

%---------------------------------------------------------------------------------------------------------------------------
\subsubsection{Potencial de Crescimento}\label{crescimentoeconomico}
O potencial de crescimento deste produto é bastante elevado, já que o mesmo é um produto inovador e que irá permitir que os
utilizadores tenham um maior controlo sobre os seus dispositivos eletrónicos, sem ser necessário comprar um extensor.
É esperado que se o produto for bem recebido por parte dos utilizadores, será necessário aumentar a produção do mesmo,
atualizar sempre que possível o software do produto, para garantir a segurança e satisfação dos utilizadores atuais e obter novos utilizadores.
Existe ainda, a chance de um investimento por parte de uma empresa maior, que pode desejar investir neste produto, e torná-lo
ainda mais conhecido e acessível a todos os utilizadores. Essa colaboração pode ser vantajosa para ambas as partes, por que a empresa maior
pode ter acesso a um produto inovador, ficando bem vista a nível mundial, e a empresa menor pode ter acesso a uma maior visibilidade e
recursos financeiros para o desenvolvimento do produto.

%-----------------------------------PAULO TRABALHO ENDS HERE-------------------------------------------------------------------

\newpage

%---------------------------------------------------------------------------------------------------------------------------
\section{Conclusão}\label{con}

Concluimos que este buraco no mercado irá permitir que o produto tenha um grande sucesso, dado ao simples facto de que a junção destas
tecnologias nunca foi realizada antes. O produto é inovador e irá permitir que o utilizador tenha um maior controlo sobre os seus dispositivos eletrónicos,
sem ser necessário comprar um extensor. Além disso, o produto é fácil de utilizar e pode ser adquirido em qualquer supermercado ou na internet,
abrindo as portas a um grande mercado.
%---------------------------------------------------------------------------------------------------------------------------

\newpage
\renewcommand{\refname}{Bibliografia} % Para artigos
\renewcommand{\bibname}{Bibliografia} % Para livros e relatórios
\addcontentsline{toc}{section}{Bibliografia} % Adiciona a Bibliografia ao índice
\printbibliography

\end{document}
